\documentclass{article}

\begin{document}
Emacs is a program that functions like a command prompt: You type
commands and it does stuff.  However, you can instruct key
combinations or menu bar selections to run emacs commands as though
you had typed them.  This makes it seem like a conventional ``GUI''
program.

Notation for key combinations: M-c = Meta c = hold down command key
and hit c.  C-x = Control x.  C-x a = Hold down control and x and the
same time, then release, and hit a.  These key combinations are case
sensitive, so C-A is different from C-a.

The way to give emacs a command is M-x <command name>.  It will then
ask for any other information it needs, and do what you asked.  This
is the 100\% general way to tell emacs to do something.  A subset of
commands have key combinations to run them, and a smaller subset have
menu options available.

Therefore there's usually more than one way to do things: You can pick
from a menu, run the command with M-x, or use a key combination.  The
way I use emacs, menu options or M-x is for things that I use
occasionally.  Key combinations are for things I use all the time.

BBDB is the ``rolodex'' program.  It doesn't have a pulldown menu.  We
can look into creating one for you if you want.

Planner is the ``To Do list'' manager.  It has pretty good pull down
menus so it should be easy to explore.  Then you can select key
combinations for the things you do a lot.

\clearpage

{\bf \Large Reference}

\begin{tabular}{lp{3in}}
{\bf \large General Emacs } \\
Exit & C-x C-c or File::Exit \\
Cancel & C-g \\
Manuals & M-x info \\
Search docs & M-x apropos \\
Enter command & M-x \\
Word wrap & 'M-x auto-fill-mode' or Options::Word Wrap in Text Modes \\
{\bf Movement} \\
Move by words & C-left and C-right\\
Move by paragraphs & C-up and C-down\\
Beginning of line & C-a \\
End of line & C-e \\
Split window & C-x 2 or C-x 3 \\
Unsplit window & C-x 1 \\
Other window & C-x o \\
Switch buffers & C-x b or C-x C-b \\
{\bf Files} \\
Find & C-x C-f or File::Open File \\
Save & C-x C-s or File::Save \\
{\bf Search \footnote{Emacs can also do regexp search and replace,
    which is a step on the road to enlightenment.}} \\
Search forward & C-s or Edit::Search::String Forward \\
Search backward & C-r or Edit::Search::String Backward \\
{\bf \large BBDB } \\
Search & M-x bbdb \\
Create & M-x bbdb-create \\
{\bf In BBDB buffer} \\
Help & h \\
Manual & i \\
Edit & e ( works on fields or records) \\ 
Delete & d (works on fields or records) \\
Abbreviate & t or *t \\
{\bf \large Planner } \\
Plan today & M-x plan \\
Create tasks & C-c t or C-c T or M-x planner-create-task, M-x
  planner-create-task-from-buffer, M-x planner-create-undated-task, M-x
  planner-create-undated-task-from-buffer \\
Calendar & M-x calendar \\
{\bf \large Contacts} \\
Add Interaction & I, then C-c C-c to save interaction to BBDB record.\\
\end{tabular}

\end{document}
